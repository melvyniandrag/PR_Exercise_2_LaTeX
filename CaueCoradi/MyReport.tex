\documentclass[11pt]{article}

% Font and color stuff for text
\usepackage{xcolor}
\usepackage{lmodern}
\usepackage[T1]{fontenc}

% Paragraph formatting
\setlength{\parindent}{0pt} 
\setlength{\parskip}{0.125cm} 

% Margins
\topmargin=-0.45in
\evensidemargin=0in
\oddsidemargin=0in
\textwidth=6.5in
\textheight=9.0in
\headsep=0.25in

% Checkboxes
\usepackage{enumitem,amssymb}
\newlist{todolist}{itemize}{2}
\setlist[todolist]{label=$\square$}

% A macro used to make the What I Know Section look "good"
\newcommand{\iknow}[2]{\par\textbf{#1} \textit{#2}}

\title{A Short Report About What I Know}
\author{ Caue Coradi }
\date{\today}

\begin{document}
\maketitle	

\section*{How to complete this assignment}
\begin{todolist}
    \item On a debian 10 droplet:
    \begin{verbatim}
        apt update
        apt install git
        apt install texlive-full
    \end{verbatim}
    \item If the droplet is new, you'll need to configure git again.
    \item Fork the repo
    \item Clone your fork to your machine ( use https or ssh, whichever one you like better )
    \item if the droplet is new and you want to use ssh, you'll need to generate a new key pair
    \item In the repo, create a directory called YourName
    \item Copy the MyReport.tex file into your directory.
    \item Use vim to fill out the "to do"s in the "What I Know" section.
    \item Change "Author" to your name.
    \item Compile using
    \begin{verbatim}
        pdflatex MyReport.tex
    \end{verbatim}
    \item If all goes well, you will see a "MyReport.pdf" in the directory!
    \item \textcolor{red}{ git add MyReport.tex}
    \item {\Large\textcolor{red}{ DO NOT git add -A or git add MyReport.pdf!}}
    \item {\LARGE\textcolor{red}{ DO NOT git add -A or git add MyReport.pdf!}}
    \item I don't want your pdf going onto github, only your .tex  file
    \item Make a pull request
    \item Use sftp to get your pdf to your home computer
    \item Submit the pdf on blackboard
    \item 50 pts for successful PR
    \item 50 pts for PDF
\end{todolist}

\section*{What I Know}

\noindent\textit{Below is a list of a few commands I know, along with brief descriptions of what they do.}
\iknow{cp}{This command is for copying files}
\iknow{cp -r}{This command is for copying entire directories}
\iknow{mv}{This command i to move files or directories to other directories}
\iknow{rm}{This command is for removing files}
\iknow{rm -r}{This command is for removing directories and its content}
\iknow{ls -a}{Displays normal and hidden files or directories}
\iknow{ls -al}{Displays all the information about files and directories, such as who owns them, who has permission to edit or read it, when it was created, etc}
\iknow{ls -l}{Displays files and directories that are not hidden}
\iknow{ssh}{Used to connect your computer to other machines}
\iknow{ssh-keygen}{Creates a new ssh key for your machine}
\iknow{cd}{Command used to navigate between the directories of your machine}
\iknow{cd ../}{Goes to the previous directory}
\iknow{cd \textasciitilde}{Takes you to the "textasciitilde" directory}
\iknow{echo}{Command to display messages in the terminal}
\iknow{cat}{Concatenates the content of a file}
\iknow{adduser}{Adds a new user to your machine}
\iknow{deluser}{Removes an user from your machine}
\iknow{chmod}{Changes the permissions users have over files} 
\iknow{chown}{Changes the owner of files}

\section*{Conclusion}
Now you know a tiny bit about using \LaTeX to prepare documents! One thing cool about \LaTeX is that you make beautiful math equations

\begin{equation}
\int_{0}^{\pi}x^2\,dx
\end{equation}

and there are ways to put images in your documents, and code snippets, and you can produce really beautifully formatted PDFs. The lecture notes for this class were all made on an Ubuntu laptop, with vim, in the terminal. And I daresay they are way more beautiful that what could have been easily made with Microsoft Word.

\end{document}
