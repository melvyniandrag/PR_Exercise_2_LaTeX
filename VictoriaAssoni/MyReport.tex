\documentclass[11pt]{article}

% Font and color stuff for text
\usepackage{xcolor}
\usepackage{lmodern}
\usepackage[T1]{fontenc}

% Paragraph formatting
\setlength{\parindent}{0pt} 
\setlength{\parskip}{0.125cm} 

% Margins
\topmargin=-0.45in
\evensidemargin=0in
\oddsidemargin=0in
\textwidth=6.5in
\textheight=9.0in
\headsep=0.25in

% Checkboxes
\usepackage{enumitem,amssymb}
\newlist{todolist}{itemize}{2}
\setlist[todolist]{label=$\square$}

% A macro used to make the What I Know Section look "good"
\newcommand{\iknow}[2]{\par\textbf{#1} \textit{#2}}

\title{A Short Report About What I Know}
\author{ Victoria Assoni }
\date{\today}

\begin{document}
\maketitle	

\section*{How to complete this assignment}
\begin{todolist}
    \item On a debian 10 droplet:
    \begin{verbatim}
        apt update
        apt install git
        apt install texlive-full
    \end{verbatim}
    \item If the droplet is new, you'll need to configure git again.
    \item Fork the repo
    \item Clone your fork to your machine ( use https or ssh, whichever one you like better )
    \item if the droplet is new and you want to use ssh, you'll need to generate a new key pair
    \item In the repo, create a directory called YourName
    \item Copy the MyReport.tex file into your directory.
    \item Use vim to fill out the "to do"s in the "What I Know" section.
    \item Change "Author" to your name.
    \item Compile using
    \begin{verbatim}
        pdflatex MyReport.tex
    \end{verbatim}
    \item If all goes well, you will see a "MyReport.pdf" in the directory!
    \item \textcolor{red}{ git add MyReport.tex}
    \item {\Large\textcolor{red}{ DO NOT git add -A or git add MyReport.pdf!}}
    \item {\LARGE\textcolor{red}{ DO NOT git add -A or git add MyReport.pdf!}}
    \item I don't want your pdf going onto github, only your .tex  file
    \item Make a pull request
    \item Use sftp to get your pdf to your home computer
    \item Submit the pdf on blackboard
    \item 50 pts for successful PR
    \item 50 pts for PDF
\end{todolist}

\section*{What I Know}

\noindent\textit{Below is a list of a few commands I know, along with brief descriptions of what they do.}
\iknow{cp}{This command is for copying files.}
\iknow{cp -r}{This command is for copying directories.}
\iknow{mv}{This command is for renaming files and directories.}
\iknow{rm}{This command is for deleting files and empty directories.}
\iknow{rm -r}{This command is for deleting directories and all its contents.}
\iknow{ls -a}{This command is for showing list of files and subdirectories of current directory, including hidden files.}
\iknow{ls -al}{This command is fow showing detailed list of files and subdirectories of current directory, including hidden files.}
\iknow{ls -l}{This command is for listing files and directories.}
\iknow{ssh}{This command is for enabling the communication of two hosts (computers).}
\iknow{ssh-keygen}{This command is for generating a public/private authentication key pair.}
\iknow{cd}{This command is for changing the current directory.}
\iknow{cd ../}{This command is for returning to the parent directory.}
\iknow{cd \textasciitilde}{This command is for navigating into the root directory.}
\iknow{echo}{This command is for displaying lines of text passed as an argument.}
\iknow{cat}{This command is for reading files and writing their output.}
\iknow{adduser}{This command is for creating a new user.}
\iknow{deluser}{This command is for deleting an existing user.}
\iknow{chmod}{This command is for changing the access permissions of files and directories.}
\iknow{chown}{This command is for changing owners of files and directories.}

\section*{Conclusion}
Now you know a tiny bit about using \LaTeX to prepare documents! One thing cool about \LaTeX is that you make beautiful math equations

\begin{equation}
\int_{0}^{\pi}x^2\,dx
\end{equation}

and there are ways to put images in your documents, and code snippets, and you can produce really beautifully formatted PDFs. The lecture notes for this class were all made on an Ubuntu laptop, with vim, in the terminal. And I daresay they are way more beautiful that what could have been easily made with Microsoft Word.

\end{document}
